\documentclass[10pt,twocolumn,letterpaper]{article}

\usepackage{cvpr}
\usepackage{times}
\usepackage{epsfig}
\usepackage{graphicx}
\usepackage{amsmath}
\usepackage{amssymb}

% Include other packages here, before hyperref.

% If you comment hyperref and then uncomment it, you should delete
% egpaper.aux before re-running latex.  (Or just hit 'q' on the first latex
% run, let it finish, and you should be clear).
\usepackage[breaklinks=true,bookmarks=false]{hyperref}

\cvprfinalcopy % *** Uncomment this line for the final submission

\def\cvprPaperID{****} % *** Enter the CVPR Paper ID heredas ein
\def\httilde{\mbox{\tt\raisebox{-.5ex}{\symbol{126}}}}

% Pages are numbered in submission mode, and unnumbered in camera-ready
%\ifcvprfinal\pagestyle{empty}\fi
%\setcounter{page}{4321}
\begin{document}

%%%%%%%%% TITLE
\title{MLCV2017: Multi-Layer Knowledge Transfer for Neural Networks}

\author{Lennard Kiehl\\
{\tt\small lennard.kiehl@gmail.com}
% For a paper whose authors are all at the same institution,
% omit the following lines up until the closing ``}''.
% Additional authors and addresses can be added with ``\and'',
% just like the second author.
% To save space, use either the email address or home page, not both
\and
Roman Remme\\
{\tt\small roman.remme@gmx.de}
}

\maketitle
%\thispagestyle{empty}

%%%%%%%%% ABSTRACT
\begin{abstract}
   The ABSTRACT is to be in fully-justified italicized text, at the top
   of the left-hand column, below the author and affiliation
   information. Use the word ``Abstract'' as the title, in 12-point
   Times, boldface type, centered relative to the column, initially
   capitalized. The abstract is to be in 10-point, single-spaced type.
   Leave two blank lines after the Abstract, then begin the main text.
   Look at previous CVPR abstracts to get a feel for style and length.
\end{abstract}

%%%%%%%%% BODY TEXT
\section{Introduction}

The idea of transferring knowledge between different architectures of neural networks, specifically from bigger models to smaller models, has been introduced in \cite{hinton2015distilling}. Part of the motivation for this process called distilling is to create a smaller model, which is faster at runtime, with the same knowledge as the bigger model. Distilling works by introducing an additional term to the loss function that links the last layers (logits) of both networks by calculating the cross entropy between them. It was shown in \cite{hinton2015distilling} that this alone serves as a very good knowledge transfer tool. We want to extend on this idea and also add links between intermediate layers. The popular VGG-16 model introduced in \cite{DBLP:journals/corr/SimonyanZ14a} serves as the bigger model and the goal is to distill each group of convolutional layers into only one convolutional layer, for an overview of the architectures see Table \ref{tab:network_architectures}. We investigate how training hyperparameters influence the process of successfully distilling knowledge.

%-------------------------------------------------------------------------
\subsection{Models}

As a big model to distill knowledge from we use VGG-16 (see \cite{DBLP:journals/corr/SimonyanZ14a}). For the smaller model that was trained with the help of the big one a similar architecture was used, where the number of convolutional layers between pooling layers was reduced from to just one. We also cut one of the fully connected layers (see Table \ref{tab:network_architectures}). The similarity made it possible to compare intermediate activations at many points in the model.

\begin{table}
	\begin{center}
	\begin{tabular}{|c|c|c|}
		\hline
		\multicolumn{3}{|c|}{Network Configurations with linkable layers} \\ \hline
			&	VGG-16	&	VGG-7	\\ \hline
			\hline
		& \multicolumn{2}{c|}{input (224$\times$224 RGB image)} \\ \hline
		link 1	&	conv3-64	&	conv3-64	\\ 
			& 	conv3-64	&		\\ \hline
		
			& \multicolumn{2}{c|}{maxpool 2$\times$2} \\ \hline
		link 2	&	conv3-128	&	conv3-128	\\ 
			& 	conv3-128	&		\\ \hline
			
			& \multicolumn{2}{c|}{maxpool 2$\times$2} \\ \hline
		link 3	&	conv3-256	&	conv3-256	\\ 
			& 	conv3-256	&		\\ 
			& 	conv3-256	&		\\ \hline
			& \multicolumn{2}{c|}{maxpool 2$\times$2} \\ \hline
		link 4	&	conv3-512	&	conv3-512	\\ 
			& 	conv3-512	&		\\ 
			& 	conv3-512	&		\\ \hline
			& \multicolumn{2}{c|}{maxpool 2$\times$2} \\ \hline
		link 5	&	conv3-512	&	conv3-512	\\ 
			& 	conv3-512	&		\\ 
			& 	conv3-512	&		\\ \hline
			& \multicolumn{2}{c|}{maxpool 2$\times$2} \\ \hline
		link 6	&	FC-4096	&	FC-4096	\\ 
			& 	FC-4096	&		\\ 
			& 	FC-10	&	FC-10	\\ \hline
			& \multicolumn{2}{c|}{softmax} \\ \hline
			
	\end{tabular}
	\end{center}
	\label{tab:network_architectures}
	\caption{\textbf{Network configurations.} The convolutional layers are denoted as conv(\textit{kernel size})-(\textit{number of channels)} and the fully 		connected layers as FC-(\textit{number of output channels}). ReLu units are omitted for brevity. The leftmost column gives the links between both networks that are added to the loss function.}
\end{table}

\subsection{Loss}
The loss function consists of three terms:
\begin{itemize}
	\item The ``hard" loss: The cross-entropy of the output distribution of the network with the correct labels
	\item The ``soft" last layer loss: The cross-entropy of the output distribution with the ``soft-targets", the output of the cumbersome model to extract knowledge from. Here, softmax layers with temperature $T$ are used.
	\item The intermediate layer loss: This is the mean squared error between the activations of pairs of layers in a certain set.
\end{itemize}
The third term is the new part of our approach. A theoretical advantage is that training times should be reduced, as gradients do not have to be propagated through the whole network to reach the first layers. Also this is a 



\subsection{Dataset}
We use CIFAR-10 \cite{krizhevsky2009learning} as the dataset to train to train both models on. It consists of 50000 training and 10000 test RGB images of size 32$\times$32 pixels. Each image belongs to one of ten classes. To use the standard VGG architecture with these low-resolution images, they are scaled up to 224$\times$224 pixels. Each image is preprocessed by subtracting the mean RGB value, computed on the training set, from each pixel. 
 

%-------------------------------------------------------------------------
\section{Training}
We used stochastic gradient descent with momentum 0.9 (CITATION NEEDED?) as an optimizer. We started with a learning rate of 0.004 and let it decay by a factor of 10 every 10 epochs for 25 epochs. 

\section{Results}
\begin{table}[]
	\begin{center}
	\begin{tabular}{|c|c|}
		\hline
		Temperature	&	Test set accuracy\\ \hline
		0.6	&	76.3 \%	\\ \hline
		1	&	76.5 \%	\\ \hline
		1.5	&	77.0 \%	\\ \hline
		2	&	\textbf{77.4} \%	\\ \hline
		2.5	&	76.7 \%	\\ \hline
		3	&	77.2 \%	\\ \hline
		5	&	73.1 \%	\\ \hline
		10	&	64.4 \%	\\ \hline
		40	&	67.3 \%	\\ \hline
	\end{tabular}
	\end{center}
	\caption{\textbf{Last layer transfer results.} bla bla}
	\label{tab:LL_results}
\end{table}

\begin{table}[]
	\begin{center}
	\begin{tabular}{|c|c|c|}
		\hline
		Linked layers	&	$\beta$	&	Test set accuracy\\ \hline
		3	&	10	&	85.9\%	\\ \hline
		2, 3, 4, 5	&	10	&	87.0\%	\\ \hline
		2, 3, 4, 5	&	40	&	87.9\%	\\ \hline
		5	&	10	&	87.7\%	\\ \hline
		3, 4	&	10	&	84.8\%	\\ \hline
		1, 2, 3, 4, 5, 6	&	10	&	81.2\%	\\ \hline
		2, 3, 4, 5, 6	&	10	&	78.5\%	\\ \hline
	\end{tabular}
	\end{center}
	\caption{\textbf{Intermediate Layers transfer results.} used last layer with temperature 2 and $ \alpha = 10 $}
	\label{tab:interemediate_results}
\end{table}


\subsection{Blind review}

Many authors misunderstand the concept of anonymizing for blind
review.  Blind review does not mean that one must remove
citations to one's own work---in fact it is often impossible to
review a paper unless the previous citations are known and
available.

Blind review means that you do not use the words ``my'' or ``our''
when citing previous work.  That is all.  (But see below for
techreports.)

Saying ``this builds on the work of Lucy Smith [1]'' does not say
that you are Lucy Smith; it says that you are building on her
work.  If you are Smith and Jones, do not say ``as we show in
[7]'', say ``as Smith and Jones show in [7]'' and at the end of the
paper, include reference 7 as you would any other cited work.

An example of a bad paper just asking to be rejected:
\begin{quote}
\begin{center}
    An analysis of the frobnicatable foo filter.
\end{center}

   In this paper we present a performance analysis of our
   previous paper [1], and show it to be inferior to all
   previously known methods.  Why the previous paper was
   accepted without this analysis is beyond me.

   [1] Removed for blind review
\end{quote}


An example of an acceptable paper:

\begin{quote}
\begin{center}
     An analysis of the frobnicatable foo filter.
\end{center}

   In this paper we present a performance analysis of the
   paper of Smith \etal [1], and show it to be inferior to
   all previously known methods.  Why the previous paper
   was accepted without this analysis is beyond me.

   [1] Smith, L and Jones, C. ``The frobnicatable foo
   filter, a fundamental contribution to human knowledge''.
   Nature 381(12), 1-213.
\end{quote}

If you are making a submission to another conference at the same time,
which covers similar or overlapping material, you may need to refer to that
submission in order to explain the differences, just as you would if you
had previously published related work.  In such cases, include the
anonymized parallel submission~\cite{Authors14} as additional material and
cite it as
\begin{quote}
[1] Authors. ``The frobnicatable foo filter'', F\&G 2014 Submission ID 324,
Supplied as additional material {\tt fg324.pdf}.
\end{quote}

Finally, you may feel you need to tell the reader that more details can be
found elsewhere, and refer them to a technical report.  For conference
submissions, the paper must stand on its own, and not {\em require} the
reviewer to go to a techreport for further details.  Thus, you may say in
the body of the paper ``further details may be found
in~\cite{Authors14b}''.  Then submit the techreport as additional material.
Again, you may not assume the reviewers will read this material. 

Sometimes your paper is about a problem which you tested using a tool which
is widely known to be restricted to a single institution.  For example,
let's say it's 1969, you have solved a key problem on the Apollo lander,
and you believe that the CVPR70 audience would like to hear about your
solution.  The work is a development of your celebrated 1968 paper entitled
``Zero-g frobnication: How being the only people in the world with access to
the Apollo lander source code makes us a wow at parties'', by Zeus \etal.

You can handle this paper like any other.  Don't write ``We show how to
improve our previous work [Anonymous, 1968].  This time we tested the
algorithm on a lunar lander [name of lander removed for blind review]''.
That would be silly, and would immediately identify the authors. Instead
write the following:
\begin{quotation}
\noindent
   We describe a system for zero-g frobnication.  This
   system is new because it handles the following cases:
   A, B.  Previous systems [Zeus et al. 1968] didn't
   handle case B properly.  Ours handles it by including
   a foo term in the bar integral.

   ...

   The proposed system was integrated with the Apollo
   lunar lander, and went all the way to the moon, don't
   you know.  It displayed the following behaviours
   which show how well we solved cases A and B: ...
\end{quotation}
As you can see, the above text follows standard scientific convention,
reads better than the first version, and does not explicitly name you as
the authors.  A reviewer might think it likely that the new paper was
written by Zeus \etal, but cannot make any decision based on that guess.
He or she would have to be sure that no other authors could have been
contracted to solve problem B.

FAQ: Are acknowledgements OK?  No.  Leave them for the final copy.


\begin{figure}[t]
\begin{center}
\fbox{\rule{0pt}{2in} \rule{0.9\linewidth}{0pt}}
   %\includegraphics[width=0.8\linewidth]{egfigure.eps}
\end{center}
   \caption{Example of caption.  It is set in Roman so that mathematics
   (always set in Roman: $B \sin A = A \sin B$) may be included without an
   ugly clash.}
\label{fig:long}
\label{fig:onecol}
\end{figure}

\subsection{Miscellaneous}

\noindent
Compare the following:\\
\begin{tabular}{ll}
 \verb'$conf_a$' &  $conf_a$ \\
 \verb'$\mathit{conf}_a$' & $\mathit{conf}_a$
\end{tabular}\\
See The \TeX book, p165.

The space after \eg, meaning ``for example'', should not be a
sentence-ending space. So \eg is correct, {\em e.g.} is not.  The provided
\verb'\eg' macro takes care of this.

When citing a multi-author paper, you may save space by using ``et alia'',
shortened to ``\etal'' (not ``{\em et.\ al.}'' as ``{\em et}'' is a complete word.)
However, use it only when there are three or more authors.  Thus, the
following is correct: ``
   Frobnication has been trendy lately.
   It was introduced by Alpher~\cite{Alpher02}, and subsequently developed by
   Alpher and Fotheringham-Smythe~\cite{Alpher03}, and Alpher \etal~\cite{Alpher04}.''

This is incorrect: ``... subsequently developed by Alpher \etal~\cite{Alpher03} ...''
because reference~\cite{Alpher03} has just two authors.  If you use the
\verb'\etal' macro provided, then you need not worry about double periods
when used at the end of a sentence as in Alpher \etal.

For this citation style, keep multiple citations in numerical (not
chronological) order, so prefer \cite{Alpher03,Alpher02,Authors14} to
\cite{Alpher02,Alpher03,Authors14}.


\begin{figure*}
\begin{center}
\fbox{\rule{0pt}{2in} \rule{.9\linewidth}{0pt}}
\end{center}
   \caption{Example of a short caption, which should be centered.}
\label{fig:short}
\end{figure*}

%------------------------------------------------------------------------
\section{Formatting your paper}

All text must be in a two-column format. The total allowable width of the
text area is $6\frac78$ inches (17.5 cm) wide by $8\frac78$ inches (22.54
cm) high. Columns are to be $3\frac14$ inches (8.25 cm) wide, with a
$\frac{5}{16}$ inch (0.8 cm) space between them. The main title (on the
first page) should begin 1.0 inch (2.54 cm) from the top edge of the
page. The second and following pages should begin 1.0 inch (2.54 cm) from
the top edge. On all pages, the bottom margin should be 1-1/8 inches (2.86
cm) from the bottom edge of the page for $8.5 \times 11$-inch paper; for A4
paper, approximately 1-5/8 inches (4.13 cm) from the bottom edge of the
page.

%-------------------------------------------------------------------------
\subsection{Margins and page numbering}

All printed material, including text, illustrations, and charts, must be kept
within a print area 6-7/8 inches (17.5 cm) wide by 8-7/8 inches (22.54 cm)
high.
Page numbers should be in footer with page numbers, centered and .75
inches from the bottom of the page and make it start at the correct page
number rather than the 4321 in the example.  To do this fine the line (around
line 23)
\begin{verbatim}
%\ifcvprfinal\pagestyle{empty}\fi
\setcounter{page}{4321}
\end{verbatim}
where the number 4321 is your assigned starting page.

Make sure the first page is numbered by commenting out the first page being
empty on line 46
\begin{verbatim}
%\thispagestyle{empty}
\end{verbatim}


%-------------------------------------------------------------------------
\subsection{Type-style and fonts}

Wherever Times is specified, Times Roman may also be used. If neither is
available on your word processor, please use the font closest in
appearance to Times to which you have access.

MAIN TITLE. Center the title 1-3/8 inches (3.49 cm) from the top edge of
the first page. The title should be in Times 14-point, boldface type.
Capitalize the first letter of nouns, pronouns, verbs, adjectives, and
adverbs; do not capitalize articles, coordinate conjunctions, or
prepositions (unless the title begins with such a word). Leave two blank
lines after the title.

AUTHOR NAME(s) and AFFILIATION(s) are to be centered beneath the title
and printed in Times 12-point, non-boldface type. This information is to
be followed by two blank lines.

The ABSTRACT and MAIN TEXT are to be in a two-column format.

MAIN TEXT. Type main text in 10-point Times, single-spaced. Do NOT use
double-spacing. All paragraphs should be indented 1 pica (approx. 1/6
inch or 0.422 cm). Make sure your text is fully justified---that is,
flush left and flush right. Please do not place any additional blank
lines between paragraphs.

Figure and table captions should be 9-point Roman type as in
Figures~\ref{fig:onecol} and~\ref{fig:short}.  Short captions should be centred.

\noindent Callouts should be 9-point Helvetica, non-boldface type.
Initially capitalize only the first word of section titles and first-,
second-, and third-order headings.

FIRST-ORDER HEADINGS. (For example, {\large \bf 1. Introduction})
should be Times 12-point boldface, initially capitalized, flush left,
with one blank line before, and one blank line after.

SECOND-ORDER HEADINGS. (For example, { \bf 1.1. Database elements})
should be Times 11-point boldface, initially capitalized, flush left,
with one blank line before, and one after. If you require a third-order
heading (we discourage it), use 10-point Times, boldface, initially
capitalized, flush left, preceded by one blank line, followed by a period
and your text on the same line.

%-------------------------------------------------------------------------
\subsection{Footnotes}

Please use footnotes\footnote {This is what a footnote looks like.  It
often distracts the reader from the main flow of the argument.} sparingly.
Indeed, try to avoid footnotes altogether and include necessary peripheral
observations in
the text (within parentheses, if you prefer, as in this sentence).  If you
wish to use a footnote, place it at the bottom of the column on the page on
which it is referenced. Use Times 8-point type, single-spaced.


%-------------------------------------------------------------------------
\subsection{References}

List and number all bibliographical references in 9-point Times,
single-spaced, at the end of your paper. When referenced in the text,
enclose the citation number in square brackets, for
example~\cite{Authors14}.  Where appropriate, include the name(s) of
editors of referenced books.

\begin{table}
\begin{center}
\begin{tabular}{|l|c|}
\hline
Method & Frobnability \\
\hline\hline
Theirs & Frumpy \\
Yours & Frobbly \\
Ours & Makes one's heart Frob\\
\hline
\end{tabular}
\end{center}
\caption{Results.   Ours is better.}
\end{table}

%-------------------------------------------------------------------------
\subsection{Illustrations, graphs, and photographs}

All graphics should be centered.  Please ensure that any point you wish to
make is resolvable in a printed copy of the paper.  Resize fonts in figures
to match the font in the body text, and choose line widths which render
effectively in print.  Many readers (and reviewers), even of an electronic
copy, will choose to print your paper in order to read it.  You cannot
insist that they do otherwise, and therefore must not assume that they can
zoom in to see tiny details on a graphic.

When placing figures in \LaTeX, it's almost always best to use
\verb+\includegraphics+, and to specify the  figure width as a multiple of
the line width as in the example below
{\small\begin{verbatim}
   \usepackage[dvips]{graphicx} ...
   \includegraphics[width=0.8\linewidth]
                   {myfile.eps}
\end{verbatim}
}


%-------------------------------------------------------------------------
\subsection{Color}

Please refer to the author guidelines on the CVPR 2017 web page for a discussion
of the use of color in your document.

%------------------------------------------------------------------------
\section{Final copy}

You must include your signed IEEE copyright release form when you submit
your finished paper. We MUST have this form before your paper can be
published in the proceedings.


{\small
\bibliographystyle{ieee}
\bibliography{egbib}
}

\end{document}
